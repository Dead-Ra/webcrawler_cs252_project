\documentclass[]{article}

\usepackage{graphicx}
\usepackage{amsmath}
\usepackage{amsthm}
\usepackage{amssymb}
\usepackage{url}
\usepackage{multirow}
\usepackage{times}
\usepackage{fullpage}

\newcommand{\comment}[1]{}

\title{CS252: An Example Project Report}
\author{
\begin{tabular}{cc}
	Student Name 1 & Student Name 2 \\
	Roll Number 1 & Roll Number 2 \\
	\url{email1@iitk.ac.in} & \url{email2@iitk.ac.in} \\
	\multicolumn{2}{c}{Indian Institute of Technology, Kanpur}
\end{tabular}
}
\date{Final report \\	% replace by ``initial'' or ``final'' as appropriate
15th April, 2013}	% replace by actual date of submission or \today

\begin{document}

\maketitle

\comment{

\begin{abstract}
	%
	Abstract of the project.
	%
\end{abstract}

}

\section{Problem Statement}

State the problem as clearly and as formally as possible.

Explain the notations, etc.

Explain the objectives, and all the inputs.

\section{Method Used}

Details of the method.

Put in a pseudo-code, etc.
Explain with figures.

\comment{

Use the following format for figures:

\begin{figure}[t]
	\centering
	\includegraphics[width=0.95\columnwidth]{figure_file}
	\caption{This figure explains this.}
	\label{fig:block}
\end{figure}

And refer as Figure \ref{fig:block}.

}

\section{Results}

Details of results, in tabular and/or graphical formats.

Describe the UI in detail.

More importantly, analyze the results.

\comment{

\begin{table}[t]
	\centering
	\begin{tabular}{|c||cc|}
		\hline
		Header 1 & Desc 1 & Desc 2 \\
		\hline
		\hline
		Row 1 & Data 1-1 & Data 1-2 \\
		Row 2 & Data 2-1 & Data 2-2 \\
		\hline
	\end{tabular}
	\caption{Table of results.}
	\label{tab:results}
\end{table}

And refer as Table \ref{tab:results}.

}

\section{Conclusions}

Clearly state the conclusions.

Outline the future work.

\section*{References}

Directly type in bib entries.

Better is to use bibtex.

\end{document}

